\hfill \break
\justifying
Apectos como los gestos, muecas, señas, etc. son factores influyentes en la comunicación cuando son interpretados, sin embargo el recurso predilecto para la comunicación humana no es otro que el elemento simbólico fundamental que llamamos lenguaje. El lenguaje humano es prueba del desarrollo intelectual de nuestra especie, al punto que logró convertirse en el contexto de la historia de la evolución humana, en la piedra angular de la que nuestra especie se ha valido como herramienta para generar conocimiento y perpetuarlo, comunicarnos entre nosotros, e incluso expresar nuestros sentimientos y pensamientos

\hfill \break
\justifying
A nivel oral, expresamos el lenguaje mediante el habla, alcanzada en algún punto de la evolución del sistema canal vocal-auditivo en el humano por el descenso de la laringe, otorgándonos la posibilidad de crear sonidos[1]. A través de la especialización como especie en la actividad del habla, se fueron otorgando semántica a los sonidos generados, permitiendo a la vez la asociación de un significado a estos[1].

\hfill \break
\justifying
Tratándose de la herramienta cotidiana que permite la comunicación e interacción social con los demás, es natural la relevancia que se le otorga al lenguaje en la vida del humano, resaltando específicamente su derivación oral como tema de interes en el estudio de este trabajo. La vivencia ordinaria diaria se dislumbra compleja cuando se restringe el uso del lenguaje verbal, y sin embargo la realidad presenta la existencia de sectores de la sociedad que a raíz de trastornos y afecciones, afrontan dificultades que parcial o completamente, les impiden expresarse oralmente.

\hfill \break
\justifying
A continuación se mencionan las afectaciones más habituales que tienen la posibilidad de beneficiarse de la implementación de la propuesta en este trabajo. Las primeras son las \textbf{afasias}[2-4], las cuales son problemas médicos originados por una lesión cerebral y que resulta en la pérdida o alteración del lenguaje.  Las \textbf{apraxias}[2,5,6] que son trastornos neurológicos caracterizado por la pérdida de la capacidad de llevar a cabo movimientos diestros y gestos, aún cuando se tenga el deseo y la habilidad física para hacerlo, teniendo diferentes afectaciones en función de la parte lesionada en el cerebro. Finalmente existe la \textbf{disartria}[2,7,8] que es un trastorno de la ejecución motora del habla debido a un problema neurológico por la presencia de un accidente u lesiones cerebrales. Afecta gravemente la motricidad de los músculos para el habla. Importante mencionar que en algunos casos los profesionales médicos recomiendan el uso de algún dispositivo electrónico o tecnológico de apoyo para la comunicación para las disartrias.[8]

\hfill \break
\justifying
Otro tipo de trastorno que imposibilita de forma indirecta la capacidad para la expresión hablada, aunque como consecuencia de la afección principal, es la sordera de percepción total y es clasificada como un trastorno de la audición. Este proyecto no considera a este sector de la población como el público objetivo principal por razones como: El amplio desarrollo del Lenguaje de Señas como principal recurso de comunicación, y las limitantes propias del trabajo donde no se provee una solución para una comunicación bidireccional con usuarios con este tipo de afección.


\hfill \break
\justifying
En este sentido y debido a la aparente utilidad e innovación en el uso de guantes de traducción para la comunicación entre personas con afectaciones en el habla y audición con la sociedad en general, se han desarrollado cantidad de trabajos a escala internacional, nacional e incluso interinstitucional basándose principalmente en el lenguaje de señas respectivo del país donde se investigó.

%
%	Estado del arte
%
\hfill \break
\justifying
Anteriormente se han desarrollados trabajos de titulación enfocados en sistemas de apoyo para la comunicación de personas con afecciones en el habla y que se encargan de realizar una traducción del LSM(Lenguaje de Señas Mexicano) al español mediante la reproducción sonora de las letras o palabras. La solución propuesta en el trabajo \textit{Guante traductor de señas para sordomudos}[9], se basa precisamente en un prototipo de guante equipado con sensores de flexión en los dedos y que son procesados y controlados mediante un $\mu C$ que enlazado a un sintetizador de voz y una pequeña pantalla LCD se reproduce el mensaje identificado; Este guante se encuentra limitado a 26 letras del abecedario y algunas abreviaturas.

\hfill \break
\justifying
Similar al trabajo anterior, pero empleando técnicas de Visión Artificial o Visión por Computadora, el trabajo "\textit{Sistema de comunicación auditiva para personas con problemas del habla}"[10], utiliza la tecnología infrarroja del dispositivo Kinect desarrollado por Microsoft para la obtención de imágenes con las que al aplicarse un algoritmo de clasificación de características en las familias de \textit{Random Forest} y \textit{Mean Shift}, son implementadas en complemento con un modelo de Redes Neuronales para obtener los resultados de las señales realizadas por el usuario.

\hfill \break
\justifying
En el ámbito internacional se han desarrollado trabajos enfocados en miniaturizar y disminuir el hardware necesario en la creación de un \textit{wearable} sensor del movimiento de las manos y los dedos en forma de una pulsera[11]. En este trabajo desarrollado por un equipo de la Universidad de Amrita en la India, utilizan en conjunto un sensor IMU(\textit{Inertial Measurement Unit}) integra un par de sensores: acelerómetro y giroscopio, junto con un arreglo de electrodos EMG(\textit{Electromyography}) que permiten reconocer la contracción de los músculos para el movimiento respectivo de cada dedo.

\hfill \break
\justifying
En otro trabajo realizado por un equipo de 3 investigadores en el \textit{National institute of Technology Puducherry Karaikal India}[12], se propone un prototipo de guante que implementa sensores de flexión, acelerómetro y giroscopio compilando sus mediciones con un Arduino Nano y que permite el envio de la recopilación vía Bluetooth a una PC que corre un algoritmo de ML para la clasificación de los gestos, SVM(\textit{Support Vector Machine}). Este prototipo además permite la identificación de gestos correspondientes al Lenguaje de Señas Americano(ASL) y el Lenguaje de Señas Indio(ISL).

\hfill \break
\justifying
Las mayoría de las propuestas de solución en los trabajos consultados tienen por tendencia el uso extendido de prototipos guantes, o incluso \textit{wearables}, con los que se realiza la detección para la interpretación, sin embargo esta clase de solución cuenta con inconvenientes principalmente físicos, como la cantidad de hardware requerida, la portabilidad del prototipo y la falta de estética para un uso cotidiano generalizado. En otra rama tecnológica pero con soluciones existentes en cantidades casi iguales, se tienen los proyectos enfocados en soluciones basadas en Visión Artificial, las cuales suelen ser mucho más cómodas en términos de portabilidad para el usuario, pero en constraste requieren de un ambiente de iluminación y constraste controlados para adquirir resultados aceptables con las técnicas de análisis de imágenes.

\hfill \break
\justifying
Tomando en cuenta estas consideraciones, la solución tecnológica propuesta en este trabajo se decanta por un prototipo físico \textit{wearable} tipo guante, con el que se busca aminorar el impacto de portabilidad y estética al utilizarse un único dispositivo sensor, el SoC micro:bit, que integra un IMU del que principalmente se utiliza para el muestreo de los patrones de movimiento el acelerómetro. Estos patrones manuales serán elementos pertenecientes a un código motriz especialmente propuesto para este sistema, y con los que se busca clasificar cada patrón en una de las clases textuales con las que se logra conformar el texto, a través de la implementación conjunta del algoritmo DTW y un modelo de Inteligencia Artificial.
Una última etapa, ejecutada después de la conformación de las palabras o frases, se logrará mediante la consulta de un servicio de \textit{Text-to-Speech} para la reproducción sonora del texto armado. El proyecto \textit{TTS de Mozilla}, es una solución que se ejecuta como servicio en red local, aunque también se considera el uso de servicios especializados en la nube, como por ejemplo \textit{Microsoft Azure Text-to-speech}.





