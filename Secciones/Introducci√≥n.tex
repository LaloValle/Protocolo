\hfill \break
\justifying
A nivel oral, expresamos el lenguaje mediante el habla, alcanzada en algún punto de la evolución del sistema canal vocal-auditivo en el humano por el descenso de la laringe, otorgándonos la posibilidad de crear sonidos[1]. A través de la especialización como especie en la actividad del habla, se fueron otorgando semántica a los sonidos generados, permitiendo a la vez la asociación de un significado a estos[1].

\hfill \break
\justifying
Se mencionan las afectaciones más habituales que tienen la posibilidad de beneficiarse de la implementación de la propuesta en este trabajo. Las primeras son las \textbf{afasias}[2-4], las cuales son problemas médicos originados por una lesión cerebral y que resulta en la pérdida o alteración del lenguaje.  Las \textbf{apraxias}[2,5,6] que son trastornos neurológicos caracterizado por la pérdida de la capacidad de llevar a cabo movimientos diestros y gestos, aún cuando se tenga el deseo y la habilidad física para hacerlo, teniendo diferentes afectaciones en función de la parte lesionada en el cerebro. Finalmente existe la \textbf{disartria}[2,7,8] que es un trastorno de la ejecución motora del habla debido a un problema neurológico por la presencia de un accidente u lesiones cerebrales. Afecta gravemente la motricidad de los músculos para el habla. Importante mencionar que en algunos casos los profesionales médicos recomiendan el uso de algún dispositivo electrónico o tecnológico de apoyo para la comunicación para las disartrias.[8]

\hfill \break
\justifying
Otro tipo de trastorno que imposibilita de forma indirecta la capacidad para la expresión hablada, aunque como consecuencia de la afección principal, es la sordera de percepción total y es clasificada como un trastorno de la audición. Este proyecto no considera a este sector de la población como el público objetivo principal por razones como: El amplio desarrollo del Lenguaje de Señas como principal recurso de comunicación, y las limitantes propias del trabajo donde no se provee una solución para una comunicación bidireccional con usuarios con este tipo de afección.


\hfill \break
\justifying
En este sentido y debido a la aparente utilidad e innovación en el uso de guantes de traducción para la comunicación entre personas con afectaciones en el habla y audición con la sociedad en general, se han desarrollado cantidad de trabajos a escala internacional, nacional e incluso interinstitucional basándose principalmente en el lenguaje de señas respectivo del país donde se investigó.

%
%	Estado del arte
%
\hfill \break
\justifying
Se desarrolló trabajos de titulación enfocados en sistemas de apoyo para la comunicación de personas con afecciones en el habla y que se encargan de realizar una traducción del LSM(Lenguaje de Señas Mexicano) al español mediante la reproducción sonora de las letras o palabras. La solución propuesta en el trabajo ”\textit{Guante traductor de señas para sordomudos}”[9]. Además en el trabajo[10] se agrega Visión Artificial con el dispositivo Kinect, en complemento con un modelo de Redes Neuronales para identificar las señas.

\hfill \break
\justifying
En el ámbito internacional se han desarrollado trabajos enfocados en miniaturizar y disminuir el hardware necesario en la creación de un \textit{wearable} sensor del movimiento de las manos y los dedos en forma de una pulsera[11]. En este trabajo desarrollado por un equipo de la Universidad de Amrita en la India, utilizan en conjunto un sensor IMU(\textit{Inertial Measurement Unit}) integra un par de sensores: acelerómetro y giroscopio, junto con un arreglo de electrodos EMG(\textit{Electromyography}) que permiten reconocer la contracción de los músculos para el movimiento respectivo de cada dedo.

\hfill \break
\justifying
En otro trabajo realizado por un equipo de 3 investigadores en el \textit{National institute of Technology Puducherry Karaikal India}[12], se propone un prototipo de guante que implementa sensores de flexión, acelerómetro y giroscopio compilando sus mediciones con un Arduino Nano y que permite el envio de la recopilación vía Bluetooth a una PC que corre un algoritmo de ML para la clasificación de los gestos, SVM(\textit{Support Vector Machine}). Este prototipo además permite la identificación de gestos correspondientes al Lenguaje de Señas Americano(ASL) y el Lenguaje de Señas Indio(ISL).

\hfill \break
\justifying
Tomando en cuenta consideraciones de características mejorables de los trabajos referenciados, para este trabajo terminal se propone un prototipo físico \textit{wearable} tipo guante, con el que se busca aminorar el impacto de portabilidad y estética al utilizarse un único dispositivo sensor, el SoC micro:bit, que integra un IMU del que principalmente se utiliza para el muestreo de los patrones de movimiento el acelerómetro. Estos patrones manuales serán elementos pertenecientes a un código motriz especialmente propuesto para este sistema, y con los que se busca clasificar cada patrón en una de las clases textuales con las que se logra conformar el texto, a través de la implementación conjunta del algoritmo DTW y un modelo de Inteligencia Artificial.
Una última etapa, ejecutada después de la conformación de las palabras o frases, se logrará mediante la consulta de un servicio de texto a voz para la reproducción sonora del texto armado. El proyecto \textit{TTS de Mozilla}, es una solución que se ejecuta como servicio en red local, aunque también se considera el uso de servicios especializados en la nube, como por ejemplo \textit{Microsoft Azure Text-to-speech}.
Dichas características no se encontraron en los trabajos referenciados.





