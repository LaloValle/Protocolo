\begin{tabular}{p{0.5cm} p{14.8cm}}
	\text{[1]} & C. J .G. Ayala Aburto, "Guante traductor de señas para sordomudos," Tésis título licenciatura, ESIME, unidad Azcapotzalco,. Ciudad de México, México, 2018. \\ \\
	
	\text{[2]} & E. D. Jiménez Carbajal, G. E. Rivera Taboada, "Sistema de comunicación auditiva para personas con problemas del habla", Tésis para título de licenciatura, ESCOM, Ciudad de México, México, 2013. \\ \\
	
	\text{[3]} & D. Vishal, H. M. Aishwarya, K. Nishkala, B. T. Royan and T. K. Ramesh, "Sign Language to Speech Conversion,"(en inglés) 2017 IEEE International Conference on Computational Intelligence and Computing Research (ICCIC), 2017, pp. 1-4, doi: 10.1109/ICCIC.2017.8523832. \\ \\
	
	\text{[4]} & M. M. Chandra, S. Rajkumar and L. S. Kumar, "Sign Languages to Speech Conversion Prototype using the SVM Classifier," TENCON 2019 - 2019 IEEE Region 10 Conference (TENCON), 2019, pp. 1803-1807, doi: 10.1109/TENCON.2019.8929356. \\ \\
	
	\text{[5]} & N. Marques, "¿Qué es el lenguaje?," \textit{Babbel}, 02, 2018 [En línea]. Disponible en https://es.babbel.com/es/magazine/que-es-lenguaje \\ \\
	
	\text{[6]} & "Presentación de Resultados," INEGI Censo 2020., México, 2020. [En línea]. Disponible en https://www.inegi.org.mx/contenidos/programas/ccpv/2020/doc/Censo2020\_Principales\_resultados\_EUM.pdf \\ \\
	
	\text{[7]} & O. Castillero Mimenza, "Los 8 tipos de trastornos del habla," \textit{Psicología y Mente}. [En línea]. Disponible en https://psicologiaymente.com/clinica/tipos-trastornos-habla \\ \\
	
	\text{[8]} & National Institute on Deafness and Other Communication Disorders,(2017, 03. 06). "La afasia".[En línea]. Disponible en https://www.nidcd.nih.gov/es/espanol/afasia \\ \\
	
	\text{[9]} & A. Triglia, "Afasias:los principales trastornos del lenguaje," \textit{Psicología y Mente}. [En línea]. Disponible en https://psicologiaymente.com/clinica/afasias-trastornos-lenguaje  \\ \\
	
	\text{[10]} & "Afasia de Broca," National Aphasia Association.[En línea]. Disponible en https://www.aphasia.org/es/afasia-de-broca/ \\ \\
	
	\text{[11]} & "Apraxia", \textit{Instituto Nacional de Trastornos Neurológicos y Accidentes Cerebrovasculares}. 03, 2022.[En línea]. Disponible en https://espanol.ninds.nih.gov/es/trastornos/apraxia \\ \\
	
	\text{[12]} & J. Huang, "Apraxia," \textit{MSD}. 10, 2021.[En línea]. Disponible en https://www.msdmanuals.com/es-mx/hogar/enfermedades-cerebrales,-medulares-y-nerviosas/disfunci\%C3\%B3n-cerebral/apraxia \\ \\
	
	\text{[13]} & "La Disartria," \textit{American Speech Language Hearing Association}.[En línea]. Disponible en https://www.asha.org/public/speech/Spanish/La-Disartria/ \\ \\
	
	\text{[14]} & J. Huang, "Disartria," \textit{MSD}. 10, 2021.[En línea]. Disponible en https://www.msdmanuals.com/es-mx/hogar/enfermedades-cerebrales,-medulares-y-nerviosas/disfunci\%C3\%B3n-cerebral/disartria \\ \\
	
	\text{[15]} &  INEGI (2020).[En línea]. Disponible en https://www.inegi.org.mx/temas/educacion/ \\ \\
	
\end{tabular}