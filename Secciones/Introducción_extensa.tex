\hfill \break
\justifying
Apectos como los gestos, muecas, sonidos, etc. son factores influyentes en la comunicación del estado consciente o inconsciente de una persona que lo hace visible para ser interpretable por los demás. Sin embargo en la comunicación humana, aunque relevantes estos aspectos, es inevitable reconocer al lenguaje como el elemento simbólico fundamental en la interacción entre individuos y grupos.

\hfill \break
\justifying
El lenguaje humano como la prueba de especímenes que han desarrollado un intelecto sofisticado, ha sido en la historia de la evolución humana la piedra angular de la que nuestra especie se ha valido como herramienta para generar conocimiento y perpetuarlo para ser compartido y adquirido por generaciones posteriores, que ha permitido la comunicación para la organización de grupos con un objetivo en común, incluso como el medio para la expresión de sentimientos y pensamientos en forma de literatura y poemas.

\hfill \break
\justifying
A nivel oral, expresamos el lenguaje mediante el habla, alcanzado en algún punto de la evolución del sistema canal vocal-auditivo en el humano, derivado del descenso de la laringe y lo que nos dio la posibilidad de crear sonidos. A través de la especialización como especie en la actividad del habla, se fueron otorgando semántica a los sonidos generados y nos permitió la asociación de un significado a estos.

\hfill \break
\justifying
Es entonces sencillo visualizar la importancia del lenguaje, resaltando especificamente el lenguaje oral como tema de estudio en este trabajo, pues es la herramienta cotidiana utilizada por la gran mayoría de las personas para la comunicación e interacción social. Contrastado por un porcentaje muy mayor de personas que no cuentan con ninguna circunstancia que le impida el habla, un sector de la población en la sociedad se ve limitado en el aprovechamiento de este recurso humano(Siendo 2,234,303 personas, un aproximado del 1.76\% de la población total en México para el Censo 2020 para personas con dificultades o discapacidad para hablar o comunicarse), originado por trastornos y afecciones que dificultan, o incluso impiden e imposibilitan, que una persona tenga la capacidad de expresarse de manera oral mediante el habla.
Entre las afecciones más habituales se mencionan 3 de ellas que podrían verse beneficiadas por el uso de la solución propuesta en este trabajo:
\begin{itemize}
	%
	%	Afasias
	%
	\item \textbf{Afasias}: Problema médico causado por la presencia de una alteración o lesión cerebral que resulta en la pérdida o alteración del lenguaje principalmente en sujetos adultos. En función de la zona encefálica en la que se tuvo el daño serán los efectos en el lenguaje, siendo en la mayoría de las ocasiones en el lóbulo izquierdo. \\ Se han identificado y estudiado una diversidad de tipos de afasias al clasificarse tomando en cuenta la presencia de fluidez verbal, commprensión verbal y capacidad de repetición en diferentes tipos de lesiones, de ahí que se mencione un par de ellas en las que principalmente la comprensión se encuentra intacta o se conserva a un buen nivel:
	\begin{itemize}
		\item \textbf{Afasia de Broca}: El área de Broca es una región localizada en el lóbulo cerebral izquierdo y está relacionada con el uso del lenguaje. Específicamente la afasia en la que se sufre un daño en esta área, tienen dificultad para la expresión fluida, la pronunciación y modulación del tono de voz. La producción de los sonidos correctos y encontrar las palabras correctas suelen ser trabajos laboriosos, sin embargo la compresión del habla de otras personas es relativamente buena, por lo que entender textos o lenguaje oral en comparación con su capacidad de hablar y escribir se encuentra mejor conservada.
		\item \textbf{Afasia motora transcortical}: Parecida a la afasia de Broca en la dificultad del paciente para la emisión de un lenguaje fluido y coherente mientras una relativa compresión de lenguaje se mantiene, con diferencia en que los sujetos con esta afección si son capaces de repetir lo que se les dice, mientras que aquellos que sufren de afasia de Broca son incapaces de repetir frases.
	\end{itemize}
	%
	%	Apraxia
	%
	\item \textbf{Apraxia:} Es un trastorno neurológico caracterizado por la pérdida de la capacidad de llevar a cabo movimientos diestros y gestos, así como de recordar patrones o secuencias de movimientos, aún cuando se tenga el deseo y la habilidad física para hacerlo. La apraxia suele originarse por lesiones debido accidentes cerebrovasculares, tumores cerebrales o traumatismo craneoencefálico, y que causan una lesión en los lóbulos paretales o en las vías nerviosas que conectan estos lóbulos con otras partes del cerebro. \\ Se identifican apraxias más comúnes como:
	\begin{itemize}
		\item \textbf{Apraxia bucofacial u orofacial}: Incapacidad de realizar movimientos faciales a voluntad, como pasar la lengua por los labios, silbar, toser o guiñar el ojo.
		\item \textbf{Apraxia Verbal}: Dificultad para coordinar los movimientos de la boca y del habla
	\end{itemize}
	%
	% 	Disartria
	%
	\item \textbf{Disartria}: Trastorno de la ejecución motora del habla, debido a un problema neurológico debido a la presencia de un accidente cerebrovascular, traumas craneoencefálicos u otras lesiones cerebrales, que causa que los músculos que emiten el habla no presenten el debido tono muscular, debilitándose con el tiempo los fibras musculares del rostros y posiblemente el sistema respiratorio. \\ Entre los síntomas se tienen el habla entrecortada jadeante, irregular, imprecisa o monótona, lenta, o rápida y "entre dientes", entonación anormal, cambios del timbre del voz, ronquera, babeo o escaces del control de la saliva y la movilidad limitada de la lengua, los labios y la mándibula.\\ Importante mencionar que en algunos casos los profesionales médicos recomiendan el uso de algún dispositivo electrónico o tecnológico de apoyo para la comunicación.
\end{itemize}

\hfill \break
\justifying
Otro tipo de trastorno que también afecta de forma indirecta a la capacidad para la expresión hablada del lenguaje como consecuencia de la afección principal, es la sordera de percepción total y es clasificado como un trastorno de la audición. Aún cuando este proyecto propuesto podría ser útil en algunas situaciones para sujetos con este trastorno, sus condiciones limitan el uso y por lo tanto no es el principal sector de la población considerada para beneficiarse del trabajo. Este tipo de afecciones requiere de consideraciones adicionales que garanticen la comunicación bidireccional entre un individuo con sordera y otro sin trastornos auditivos, esto con el objetivo de proponer una tecnología atractiva y útil para su uso por el público objetivo.

\hfill \break
\justifying
Frente a esta problemática, se propone en este documento el desarrollo de una solución tecnológica basada en el SoC micro:bit, que permita el sensado del movimiento de una mano utilizando el acelerometro integrado, y que a partir de unos patrones de movimientos predefinidos con un sencillo lenguaje propuesto, se puedan identificar letras del alfabeto en español como interfaz para la reproducción sonora de los fonemas en el idioma español de México y su conformación en sílabas y finalmente palabras completas.
Para conseguir esto a partir de los patrones de movimiento del lenguaje propuesto, se identifican y clasifican los movimientos que son traducidos a una cadena de texto y que finalmente para su reproducción sonora se utiliza un servicio de \textit{Text-to-Speech}, como el proyecto \textit{TTS de Mozilla} y servicios especializados en la nube como \textit{Microsoft Azure Text-to-speech}.

\hfill \break
\justifying
El alfabeto fonético internarcional es un sistema de notación fonética creado por lingüistas con el proposito de establecer de forma regularizada, precisa y única, la representación de los sonidos del habla de cualquier lengua. Bajo este esquema alfabético, el español se caracteriza por contar con 22 fonemas descritos en este, y que se dividen en 19 consonánticos y 5 vocálicos.

\hfill \break
\justifying
Los 5 vocálicos corresponden a los fonemas /a/, /e/, /i/, /o/, /u/.

\hfill \break
\justifying
Los fonémas consonánticos a pesar de contabilizarse 19, en realidad son utlizados en el español de México únicamente 17, siendo estos 2 fonémas de uso común únicamente en regiones específicas: /\includegraphics[width=6pt]{Imagenes/fricativa_dental_sorda.png}/ (también comunmente utilizado en el idioma inglés) y /\includegraphics[width=6pt]{Imagenes/aproximante_lateral_palatal.png}/ (también comunmente utilizado en el idioma italiano).

\begin{figure}[!h]
	\centering
	\begin{tabular}{c | p{5.5cm} | p{4.5cm}}
		Fonema 	& \hfil Orden \hfil 						& \hfil Ortográfico \hfil \\
				&								& \\ \hline
				&								& \\
		/m/		& Nasal Bilabial Sonora				& m \\
		/n/		& Nasal Alveolar Sonora				& n \\
		/\includegraphics[width=5pt]{Imagenes/nasal_palatal.png}/		& Nasal Palatar Sonora				& ñ \\
		/p/		& Oclusiva Bilabial Sorda			& p \\
		/b/		& Oclusiva Bilabial Sonora			& b \\
		/t/		& Oclusiva Dental Sorda				& t \\
		/d/		& Oclusiva Dental Sonora			& d \\
		/k/		& Oclusiva Velar Sorda				& c+a,o,u ; qu+e,i ; k \\
		/g/		& Oclusiva Velar Sonora				& g+a,o,u ; gu+e,i \\
		/f/		& Fricativa Labiodental Sorda		& f \\
		/s/		& Fricativa Alveolar Sorda			& s \\
		/\includegraphics[width=7pt]{Imagenes/fricativa_palatal_sonora.png}/		& Fricativa Palatal Sonora			& y ; hi+vocal ; ll \\
		/x/		& Fricativa Velar Sorda				& j+a,e,i,o,u ; g+e,i \\
		/\includegraphics[width=6pt]{Imagenes/africada_postalveolar_sorda.png}/		& Africada Postalveolar Sorda		& ch \\
		/r/		& Vibrante Múltiple Alveolar Sonora	& rr ; r-(posición inicial o precedida de l, n, s) \\
		/\includegraphics[width=5pt]{Imagenes/vibrante_alveolar_simple.png}/		& Vibrante Simple Alveolar Sonora	& r(no inicial ni precedidad de n, l, s) \\
		/l/		& Aproximante Lateral Alveolar Sonora& l
	\end{tabular}
	\label{Tabla_fonemas}
	\caption{Los 17 fonémas básicos considerados en la implementación del sistema para su reproducción sonora}
\end{figure}

\hfill\break
\hfill\break
\justifying
El idioma español a su vez como lenguaje escrito, cuenta con un total de 27 símbolos alfabéticos utilizados para la escritura de las palabras, además de 3 compuestos por 2 letras combinadas(rr, ch, ll) que también cuentan con un fonema específico asociado. Aún cuando no es homogeneamente aplicable la asociación de un fonema a cada letra, al menos 11 de los fonemas totales usados en español, tienen una correspondencia directa con una letra y el ch, cuando estas son pronunciadas oralmente.
\hfill\break
\justifying
Aunado a estas correspondencias, se hace notar que el símbolo h, letra muda en español, no requiere representación como patrón dentro del sencillo lenguaje motriz que se desarrollará, así como también se omitirá la asociación directa de algunas letras por multiple fonética contextual a la posición y combinación de otras letras en las palabras.
\hfill\break
\hfill\break
\justifying
De forma intuitiva para la mayor parte de la población(cerca del 95\% de la población que es alfabeta en México) la asociación de la pronunciación de las letras en las palabras, es natural mediante la relación en su representación escrita, siendo mucho más complejo un desglose directo en los fonémas que la compone, inclusive podría describirse como imposible por el desconocimiento del colectivo general de una representación estandarizada de los sonidos, marco que es propuesto por el Alfabeto Fonético Internacional.
\hfill\break
\justifying
Con este hecho en mente, la ortografía de las palabras será utilizada como interfaz para describir la palabra, o directamente los sonidos, que el usuario desea expresar.
\hfill\break
\justifying
La creación de un sencillo lenguaje codificado en un contexto de patrones de movimiento, le servirá al usuario para indicar el sonido que desea sea reproducido.

\hfill\break
\justifying
Probablemente el referente más cercano e inmediato a este sencillo lenguaje con el que podría ser comparado, es el LSM(Lenguaje de Señas Mexicano), sin embargo, y aún cuando este trabajo tiene el objetivo común de permitir la comunicación principalmente de una comunidad con impedimento del habla, difiere primordialmente en algunos aspectos: 

\begin{itemize}
	\item El LSM es un lenguaje que requiere para su expresión el involucramiento de gestos con movimiento desde encima de la cabeza y hasta debajo de la cadera, incluyendo movimiento de manos, expresiones faciales y mirada intencional. Por su parte el lenguaje motriz que se propondrá se limita únicamente a 1 mano, donde se colocará el SoC para el sensado.
	
	\item La interpretación de ambos lenguajes se realiza en dominios sensoriales humanos distintos. Completamente visual para el LSM, y parcialmente visual para el lenguaje propuesto, pero principalmente auditivo.
	
	\item Otra diferencia importante se encuentra en la complejidad y relación con el idioma español. Mientras el LSM cuenta con una rica y compleja gramática, y vocabulario abundante, también existe desacoplado de la estructura, gramática y normas propias del español. El enfoque principal del lenguaje que se propondrá, es la expresión de palabras y letras del español, sobrepasando sus capacidades la formulación de normas gramaticales o vocabulario propio, fungiendo únicamente como un simplificado codificado motriz para la expresión sonora del español.
	
	\item Finalmente su implementación, aún cuando es común para las personas con impedimento del habla u afecciones relacionadas a la expresión oral, el LSM brinda mayores posibilidades para la población sordo-muda. Mientras tanto el prototipo de este trabajo podrá ser poco beneficioso para la población sordo-muda frente al LSM, pero propone una alternativa para la comunicación(dentro de los límites de este trabajo, aún unilateral) entre personas con impedimento del habla hacia la población con afecciones relacionadas a la percepción visual.
\end{itemize}






